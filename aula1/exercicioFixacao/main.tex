%        File: main.tex
%     Created: sáb fev 27 02:00  2021 -0
% Last Change: sáb fev 27 02:00  2021 -0
%
\documentclass[a4paper, 12pt]{letter}

\usepackage[utf8x]{inputenc}
\usepackage[margin=2cm]{geometry}
\usepackage[brazil]{babel}
\usepackage{graphicx}


\begin{document}
UnB - Universidade de Brasilia \\
FGA - Faculdade do Gama \\
OO - Orientação por Objetos \\

\begin{center}
    \huge
    Lista de exercícios - Revisão de programação
    \normalsize
\end{center}

\begin{center}
    \textbf{Instruções para os exercícios:}
\end{center}

\begin{itemize}
  \item Esses exercícios são para fixação do conteúdo e não têm impacto no
    cálculo da menção dos alunos. 
  \item Alunos que não estiveram presentes na aula online (via MS-Teams) deverão
    entregar os exercícios, via \textit{pull-request}, no repositório da
    disciplina. Essa entrega será utilizada para que a presença na aula seja
    contabilizada. 
  \item Bons estudos!

\end{itemize}

\vspace{0.8cm}

\begin{enumerate}

  \item \textbf{Entrada e saída -- Hello World:} Crie um programa que seja capaz
    de ler do teclado, via terminal (entrada textual), o nome completo e o ano
    de nascimento de uma pessoa e ao final imprima uma mensagem informando o
    nome e a idade da pessoa. 

    Exemplo: \\
    nome: ``Maria Celia'' \\
    ano de nascimento: ``2000'' \\
    Saída: ``Olá Maria Celia!  Você possui 21 anos de idade.''

  \item \textbf{Entrada e saída gráfica} Crie uma nova versão do programa criado
    para o exercício 1 em que a entrada e saída dos dados seja feita através de
    elementos gráficos da biblioteca \texttt{JOptionPane}. Lembre-se: valores
    lidos do teclado usando essa biblioteca são sempre \texttt{Strings}. No caso
    de valores númericos, esses devem ser convertidos (\textit{cast}) para o
    tipo de dado apropriado (\texttt{int}, \texttt{float}, \texttt{double},
    etc~\ldots).

  \item \textbf{Estruturas de decisão -- if-else, switch-case:} Crie um programa
    que seja capaz de calcular o perímetro e a área das figuras geométricas
    círculo, triângulo, quadrado e retângulo. Inicialmente o programa deve
    solicitar ao usuário para qual figura geométrica ele deseja informar os
    dados. Em seguida os dados da figura geométrica escolhida deverão ser
    informados pelo usuário pelo teclado. Exemplo: para círculo deve ser
    informado apenas o valor do raio, para triângulos devem ser informados os
    valores dos três lados, e de modo similar para as demais figuras
    geométricas. Os valores dos dados deverão ser testados para que valores
    inválidos (negativos ou zerados) não sejam informados. Sugestão: utilize
    switch-case para a seleção da figura geométrica.

    Exemplo:   \\
    Figura geométrica: triângulo. \\
    Lado 1: 3  \\
    Lado 2: -4 \\
    Lado 2: 0  \\
    Lado 2: 4  \\
    Lado 3: 5  \\

    Perímetro = 12

  \item \textbf{Estruturas de repetição -- for, while, do-while:} Crie três
    versões para um mesmo programa de cálculo dos termos e da soma dos termos de
    uma progressão aritmética (PA) para os valores de $n$, $a_0$ e $r$
    informados pelo usuário. Cada versão do programa utilizará uma das
    estruturas de repetição informadas no enunciado dessa questão:

    Exemplo: \\
    1o. termo ($a_0$) = 2 \\
    Razão ($r$) = 3 \\
    Número de termos ($n$) = 5 \\

    Termos da PA = 2, 5, 8, 11, 14. \\
    Soma dos termos da PA = 40.

  \item \textbf{Vetores:} Altere o programa criado no exercício anterior de modo
    a utilizar vetores na solução do problema. Nesse caso o valor lido para $n$
    deverá ser utilizado para criação do vetor de inteiros a receber os valores
    da PA. Em um primeiro momento o programa deverá calcular cada valor da PA e
    alocá-lo em sua posição correspondente nesse vetor. Em seguida o programa
    deverá ``varrer'' o vetor para apresentar os valores e calcular a soma dos
    termos. 
    

  \item \textbf{Matrizes:} Crie um programa que leia as dimensões $m$ e $n$,
    sendo $m$ o número de linhas e $n$ o número de colunas, de duas matrizes de
    inteiros . Assim que as dimensões das matrizes forem informadas pelo
    usuário, realize a leitura de cada um dos valores das matrizes. Imprima as
    matrizes. 

  \item \textbf{Métodos (funções e procedimentos):} Altere o programa do
    exercício anterior de modo a adicionar três novos métodos. 
    \begin{itemize}
      \item crie um método chamado \texttt{calcularDeterminante} que recebe como
	parâmetro uma matriz e retorne o valor da determinante da matriz (um
	valor inteiro).
      \item crie um método chamado \texttt{somarMatrizes} que recebe duas
	matrizes como parâmetros e retorne uma terceira matriz, formada pela
	soma das matrizes. Esse método deve ser capaz de identificar se a soma
	das matrizes é possível de ser realizada.
      \item Crie um método chamado \texttt{multiplicarMatrizes} que recebe duas
	matrizes como parâmetros e retorne uma terceira matriz, formada pela
	multiplicação das matrizes. Esse método deve ser capaz de identificar se
	a multiplicação das matrizes é possível de ser realizada. 
    \end{itemize}
\end{enumerate}

\end{document}


